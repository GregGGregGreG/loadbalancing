\documentclass[11pt, a4paper]{article}
% \usepackage[T1]{fontenc}
\usepackage[utf8]{inputenc}
\usepackage{listings}
\usepackage[margin=1.0in]{geometry}
\usepackage{color}
\usepackage{graphicx}
\usepackage{tabularx}
\usepackage{url} 
\usepackage{float}
\usepackage{enumitem}
\usepackage{subcaption}

\lstnewenvironment{python}[1][]
{\lstset{
		language=python,
		numbers=left,
		numberstyle=\tiny,
		keywordstyle=\color{blue},       	% keyword style
		basicstyle=\footnotesize\ttfamily,	% font
		numbers=left,						% line numbers
		tabsize=2,							% tabsize in spaces
		#1}}
{}

\lstnewenvironment{bash}[1][]
{\lstset{
		language=bash,
		numbers=left,
		numberstyle=\tiny,
		numbers=none,
		basicstyle=\footnotesize\ttfamily,	% font
		tabsize=2,							% tabsize in spaces
		#1}}
{}

\lstnewenvironment{code}[1][]
{\lstset{numbers=left,
		numberstyle=\tiny,
		keywordstyle=\color{blue},       % keyword style
		basicstyle=\footnotesize\ttfamily,	% font
		numbers=left,						% line numbers
		tabsize=2,							% tabsize in spaces
		#1}}
{}

\title{DezSys03 - Synchronization of heterogeneous databases}
\author{Elias Frantar, Gary Ye (5AHITT)}
\date{\today{}, Wien}
\begin{document}

\lstset{
  backgroundcolor=\color{white},   % choose the background color; you must add \usepackage{color} or \usepackage{xcolor}
  basicstyle=\footnotesize,        % the size of the fonts that are used for the code
  breakatwhitespace=false,         % sets if automatic breaks should only happen at whitespace
  breaklines=true,                 % sets automatic line breaking
  captionpos=b,                    % sets the caption-position to bottom
% commentstyle=\color{mygreen},    % comment style
  deletekeywords={...},            % if you want to delete keywords from the given language
  escapeinside={\%*}{*)},          % if you want to add LaTeX within your code
  extendedchars=true,              % lets you use non-ASCII characters; for 8-bits encodings only, does not work with UTF-8
% frame=single,                    % adds a frame around the code
  keepspaces=true,                 % keeps spaces in text, useful for keeping indentation of code (possibly needs columns=flexible)
  keywordstyle=\color{blue},       % keyword style
% language=bash,                   % the language of the code
  morekeywords={*,...},            % if you want to add more keywords to the set
  numbers=none,                    % where to put the line-numbers; possible values are (none, left, right)
  numbersep=5pt,                   % how far the line-numbers are from the code
%  numberstyle=\tiny\color{gray}, % the style that is used for the line-numbers
  rulecolor=\color{black},         % if not set, the frame-color may be changed on line-breaks within not-black text (e.g. comments (green here))
  showspaces=false,                % show spaces everywhere adding particular underscores; it overrides 'showstringspaces'
  showstringspaces=false,          % underline spaces within strings only
  showtabs=false,                  % show tabs within strings adding particular underscores
%  stepnumber=1,                    % the step between two line-numbers. If it's 1, each line will be numbered
  stringstyle=\color{red},     % string literal style
  tabsize=2,                       % sets default tabsize to 2 spaces
  title=\lstname,                   % show the filename of files included with \lstinputlisting; also try caption instead of title
  belowskip=-3em,    
}

\setlength\parindent{0pt}

\maketitle
\newpage
\tableofcontents
\newpage

\section{Requirements}

\subsection{Aufgabenstellung}

Es soll ein Load Balancer mit mindestens 2 unterschiedlichen Load-Balancing Methoden (jeweils 7 Punkte) implementiert werden (ähnlich dem PI Beispiel [1]; Lösung zum Teil veraltet [2]). Eine Kombination von mehreren Methoden ist möglich. Die Berechnung bzw. das Service ist frei wählbar!
\\\\
Folgende Load Balancing Methoden stehen zur Auswahl:

\begin{itemize}
	\item Weighted Round-Robin
	\item Least Connection
	\item Least Connected Slow- Start Time
	\item Weighted Least Connection
	\item Agent Based Adaptive Balancing / Server Probes
\end{itemize}

Um die Komplexität zu steigern, soll zusätzlich eine "Session Persistence" (2 Punkte) implementiert werden.

\subsection{Tests}

Die Tests sollen so aufgebaut sein, dass in der Gruppe jedes Mitglied mehrere Server fahren und ein Gruppenmitglied mehrere Anfragen an den Load Balancer stellen. Für die Abnahme wird empfohlen, dass jeder Server eine Ausgabe mit entsprechenden Informationen ausgibt, damit die Verteilung der Anfragen demonstriert werden kann.

\subsection{Modalitäten}

Gruppenarbeit: 2 Personen
Abgabe: Protokoll mit Designüberlegungen / Umsetzung / Testszenarien, Sourcecode (mit allen notwendigen Bibliotheken), Java-Doc, Jar
\\\\
Viel Erfolg!


\subsection{Quellen}

[1] "Praktische Arbeit 2 zur Vorlesung 'Verteilte Systeme' ETH Zürich, SS 2002", Prof.Dr.B.Plattner, übernommen von Prof.Dr.F.Mattern (http://www.tik.ee.ethz.ch/tik/education/lectures/VS/SS02/Praktikum/aufgabe2.pdf)
[2] http://www.tik.ee.ethz.ch/education/lectures/VS/SS02/Praktikum/loesung2.zip

\newpage

\section{Effort Estimation and Work Distribution}

The following table compares the estimated with the actually needed amount of time for completing each individual task.

The estimation is in the second column while the columns that follow are actual values.

\parskip 12pt
\begin{tabular} {| l | c | c | c | c |}
	\hline
	Task					&	Estimation		& 	Elias 	& 	Gary 	& 	Team	\\ \hline \hline
	Preparation				&	2				&			&  			&			\\ \hline
	Design					&	4				&			&			&			\\ \hline
	Implementation			&	6				&			&			& 	 		\\ \hline
	Testing					&	4				&			& 	 		& 	 		\\ \hline
	Documentation			&	10				&			&	 		& 			\\ \hline 
	Total					&	26				&			&			& 			\\
	\hline
\end{tabular}

\section{Design}

\subsection{Assumptions}

Because of the limited time for exercise, it is not possible to handle all problems. Therefore the following assumptions have been made under which our solution will work:

\vspace{-10pt}
\begin{itemize}
	\item Connections do not end while the load-balancer is running
	\item Servers are registered at the load-balancer before and only before startup
	\item Sessions cannot end
	\item All connections have the same size/weight/package-length/...
\end{itemize}
\vspace{-10pt}

Deduced from this assumptions our implementation will have the following functional restrictions:

\vspace{-10pt}
\begin{itemize}
	\item Additional servers \textit{cannot} be dynamically added during runtime
	\item Load-balancer will \textit{not} be acknowledged when a connection is closed
	\item A client which connected to a server once will (session persistence) \textit{always} be reconnected to same server (as long as the load-balancer is up)
	\item Load will only be balanced depending on \textit{the number of connections} ignoring their length/weight/traffic/...
\end{itemize}
\vspace{-10pt}

\subsection{Technologies}

\subsection{Software Design}

\section{Ocurred Problems}


\nocite{*}
\bibliographystyle{plain}
\bibliography{bibliography}

\end{document}
